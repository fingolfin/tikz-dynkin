\documentclass[a4paper,10pt]{scrartcl}
\usepackage[margin=2cm]{geometry}
\usepackage{multicol}


\usepackage{tikz}
\usetikzlibrary{arrows,decorations.markings}

\usetikzlibrary{matrix}
\usetikzlibrary{graphs}
\usetikzlibrary{backgrounds}

\begin{document}


% Max' Tikz styles for dynkin diagrams
\def\DynkinNodeSize{3.5mm}
\def\DynkinArrowLength{3mm}
\tikzset{
  % a diagram node
  dnode/.style={
    circle,
    inner sep=0pt,
    minimum size=\DynkinNodeSize,
    fill=white,
    draw},
  %
  middlearrow/.style={
    decoration={markings,
      mark=at position 0.6 with
      %{\arrow[black]{angle 90};}
      %{\arrow[black]{angle 60};}
      %{\arrow[black]{stealth};}
      {\draw (0:0mm) -- +(+135:\DynkinArrowLength); \draw (0:0mm) -- +(-135:\DynkinArrowLength);},
    },
    postaction={decorate}
  },
  %
  leftrightarrow/.style={
    decoration={markings,
      mark=at position 0.999 with
      {
      \draw (0:0mm) -- +(+135:\DynkinArrowLength); \draw (0:0mm) -- +(-135:\DynkinArrowLength);
      },
      mark=at position 0.001 with
      {
      \draw (0:0mm) -- +(+45:\DynkinArrowLength); \draw (0:0mm) -- +(-45:\DynkinArrowLength);
      },
    },
    postaction={decorate}
  },
  % single edge
  sedge/.style={
  },
  % directed double edge
  dedge/.style={
    middlearrow,
    double distance=0.5mm,
  },
  % directed triple edge
  tedge/.style={
    middlearrow,
    double distance=1.0mm+\pgflinewidth,
    postaction={draw}, % third line
  },
  % double edge with two arrows, for \tilde{A}_1 residues
  infedge/.style={
    leftrightarrow,
    double distance=0.5mm,
  },
}

%
%
%
\section{Random example}

This example show cases most options: 
single edges,
(directed) multi edges, colored nodes, double nodes, node labels, edge labels.

\begin{center}
  \begin{tikzpicture}
    \node[dnode,label=below:$A$] (A) at (0,0) {};
    \node[dnode,label=below:$B$] (B) at (1,0) {};
    \node[dnode] (C) at (2,0) {};
    \node[dnode,fill=green] (D) at (3,0) {};
    \node[dnode,double] (E) at (2,2) {};
    \node[dnode,fill=black] (F) at (4,3) {};
    \node[dnode,label=above:{a long label}] (G) at (1,3) {};
    \node[dnode] (H) at (6,2) {};

    \path (A) edge[sedge] (B)
 		      edge[sedge] node[above left] {$\infty$} (G)
          (B) edge[dedge] (C)
          (C) edge[tedge] (D)
          (D) edge[tedge] (E)
          (F) edge[tedge] (E)
          (F) edge[sedge,dashed] (H);
  \end{tikzpicture}
\end{center}


%
%
%
\section{Spherical diagrams}

Here are the usual spherical diagrams with Bourbaki labelling:

\begin{multicols}{2}

\begin{tikzpicture}%[every node/.style={dnode}]
    \draw (-1,0) node[anchor=east]  {$A_n$};

    \node[dnode,label=below:$1$] (1) at (0,0) {};
    \node[dnode,label=below:$2$] (2) at (1,0) {};
    \node[dnode,label=below:$n-2$] (3) at (3,0) {};
    \node[dnode,label=below:$n-1$] (4) at (4,0) {};
    \node[dnode,label=below:$n$] (5) at (5,0) {};

    \path (1) edge[sedge] (2)
          (2) edge[sedge,dashed] (3)
          (3) edge[sedge] (4)
          (4) edge[sedge] (5)
          ;
\end{tikzpicture}

\ 

\begin{tikzpicture}%[every node/.style={dnode}]
    \draw (-1,0) node[anchor=east]  {$B_n$};

    \node[dnode,label=below:$1$] (1) at (0,0) {};
    \node[dnode,label=below:$2$] (2) at (1,0) {};
    \node[dnode,label=below:$n-2$] (3) at (3,0) {};
    \node[dnode,label=below:$n-1$] (4) at (4,0) {};
    \node[dnode,label=below:$n$] (5) at (5,0) {};

    \path (1) edge[sedge] (2)
          (2) edge[sedge,dashed] (3)
          (3) edge[sedge] (4)
          (4) edge[dedge] (5)
          ;
\end{tikzpicture}

\ 

\begin{tikzpicture}%[every node/.style={dnode}]
    \draw (-1,0) node[anchor=east]  {$C_n$};

    \node[dnode,label=below:$1$] (1) at (0,0) {};
    \node[dnode,label=below:$2$] (2) at (1,0) {};
    \node[dnode,label=below:$n-2$] (3) at (3,0) {};
    \node[dnode,label=below:$n-1$] (4) at (4,0) {};
    \node[dnode,label=below:$n$] (5) at (5,0) {};

    \path (1) edge[sedge] (2)
          (2) edge[sedge,dashed] (3)
          (3) edge[sedge] (4)
          (5) edge[dedge] (4)
          ;
\end{tikzpicture}

\ 

\begin{tikzpicture}%[every node/.style={dnode}]
    \draw (-1,0) node[anchor=east]  {$D_n$};

    \node[dnode,label=below:$1$] (1) at (0,0) {};
    \node[dnode,label=below:$2$] (2) at (1,0) {};
    \node[dnode,label=below:$n-3$] (3) at (3,0) {};
    \node[dnode,label=below:$n-2$] (4) at (4,0) {};
    \node[dnode,label=above:$n-1$] (5) at (5,0.5) {};
    \node[dnode,label=below:$n$] (6) at (5,-0.5) {};

    \path (1) edge[sedge] (2)
          (2) edge[sedge,dashed] (3)
          (3) edge[sedge] (4)
          (4) edge[sedge] (5)
              edge[sedge] (6)
          ;
\end{tikzpicture}

\ 

\begin{tikzpicture} %[every node/.style={dnode}]
    \draw (-1,0.5) node[anchor=east]  {$E_n$};

    \node[dnode,label=below:$1$] (1) at (0,0) {};
    \node[dnode,label=above:$2$] (2) at (2,1) {};
    \node[dnode,label=below:$3$] (3) at (1,0) {};
    \node[dnode,label=below:$4$] (4) at (2,0) {};
    \node[dnode,label=below:$5$] (5) at (3,0) {};
    \node[dnode,label=below:$6$] (6) at (4,0) {};
    \node[dnode,label=below:$n$] (n) at (5,0) {};

    \path (1) edge[sedge] (3)
          (3) edge[sedge] (4)
          (4) edge[sedge] (5)
 		      edge[sedge] (2)
          (5) edge[sedge] (6)
          (6) edge[sedge,dashed] (n);
\end{tikzpicture}

\ 

\begin{tikzpicture}%[every node/.style={dnode}]
    \draw (-1,0) node[anchor=east]  {$F_4$};

    \node[dnode,label=below:$1$] (1) at (0,0) {};
    \node[dnode,label=below:$2$] (2) at (1,0) {};
    \node[dnode,label=below:$3$] (3) at (2,0) {};
    \node[dnode,label=below:$4$] (4) at (3,0) {};

    \path (1) edge[sedge] (2)
          (2) edge[dedge] (3)
          (3) edge[sedge] (4)
          ;
\end{tikzpicture}

\ 

\begin{tikzpicture}%[every node/.style={dnode}]
    \draw (-1,0) node[anchor=east]  {$G_2$};

    \node[dnode,label=below:$1$] (1) at (0,0) {};
    \node[dnode,label=below:$2$] (2) at (1,0) {};

    \path (2) edge[tedge] (1)
          ;
\end{tikzpicture}


\end{multicols}


%
%
%
\section{Affine diagrams}

\begin{multicols}{2}

\begin{tikzpicture}%[every node/.style={dnode}]
    \draw (-1,0) node[anchor=east]  {$\tilde{A}_1$};

    \node[dnode] (1) at (0,0) {};
    \node[dnode] (2) at (1,0) {};

    \path (2) edge[sedge] node[above] {$\infty$} (1)
          ;
\end{tikzpicture}

\begin{tikzpicture}%[every node/.style={dnode}]
    \draw (-1,0) node[anchor=east]  {$\tilde{A}_1$};

    \node[dnode] (1) at (0,0) {};
    \node[dnode] (2) at (1,0) {};

    \path (2) edge[infedge] (1)
          ;
\end{tikzpicture}

\end{multicols}


%
%
%
\section{TODO}

\begin{itemize}
\item make size of nodes configurable (needs tweaked middlearrow macro)
\item even better: try to make it compatible with the ``scale=XYZ'' attribute
\item perhaps base the sizes on the active font size?
\item draw edges using \verb|\begin{pgfonlayer}{background} ... \end{...}|  to avoid overdraw artifacts
\item add more node styles: black; arbitrary color; perhaps a ``double circle
\item make size more flexible
\item allow for undirected double/triple nodes
\item Add macros for standard spherical / affine diagrams: both version for specific $n$ (e.g. for drawing
$A_3$), and also the ones we show above, with unspecified $n$
\item some people like to represent infinity edges by ``double arrows'', i.e. $\Leftrightarrow$
\item Some people prefer double/triple edges where the outer two edges are tangential to the nodes
\item Improve triple edges: right now, there is a tiny gap between the outer two edges and the node circles.
Close that.
\item Add examples which use other ways of placing nodes, e.g. a matrix
\item another approach to arrows: draw them more like normal arrows, with the tip directly at the target node.
\item another approach to arrows: draw them above the edges (and for infinity edges, draw one above and one below, pointing in opposite direction)
\item ...
\end{itemize}

\subsection*{Alternate examples for linear diagrams}


\begin{tikzpicture}
    \draw (-1,0) node[anchor=east]  {$X_n$};
    
    % \dynkinnode[label=below:$1$] (a) at (0,2);
    % \dynkinnode[label=below:$2$] (b) at (1,2);
    
    % Hypothetical syntax for purely linear, horizontal diagrams:
    
    %\dynkin  dnode[label=below:$1$]
    %         edge[triple,reversed]
    %         dnode[fill=green,label=below:$2$]
    %         edge[dashed,long]
    %         dnode
    %         edge
    %         dnode[label=below:$n-1$]
    %         edge[double]
    %         dnode[label=below:$n$]
    
    %
    % Another idea: first name some coordinates, then
    % place circle nodes on them, and then specify edges between them
    %
    \node[inner sep=0pt] (a) at (0,1) {};
    \draw (a) node[dnode,label=below:$1$] {};

    \node[inner sep=0pt] (b) at (1,1) {};
    \draw (b) node[dnode,fill=green,label=below:$2$] {};

    \node[inner sep=0pt] (c) at (3,1) {};
    \draw (c) node[dnode] {};

    \node[inner sep=0pt] (d) at (4,1) {};
    \draw (d) node[dnode,label=below:$n-1$] {};

    \node[inner sep=0pt] (e) at (5,1) {};
    \draw (e) node[dnode,label=below:$n$] {};

    \begin{scope}[on background layer]
    \path (b) edge[tedge] (a)
          (b) edge[sedge,dashed] (c)
          (c) edge[sedge] (d)
          (d) edge[dedge] (e)
          ;
    \end{scope}



%    \draw (0,3) [tedge] -- ++(1,0);
%    \draw (0,3) node[dnode,label=below:$1$] {};
%
%    \draw (0,2) node[dnode,label=below:$1$] {}
%          [tedge] -- ++(1,0) node[dnode,label=below:$2$] {};
%    
%    \draw (0,1) node[dnode,label=below:$1$] {}
%          [tedge] -- ++(1,0) node[dnode,label=below:$2$] {} 
%          {[sedge, dashed] -- ++(2,0)} node[dnode] {}
%          {[sedge] -- ++(1,0)} node[dnode,label=below:$n-1$] {}
%          {[dedge] -- ++(1,0)} node[dnode,label=below:$n$] {} ;

    \node[dnode,label=below:$1$] (1) at (0,0) {};
    \node[dnode,fill=green,label=below:$2$] (2) at (1,0) {};
    \node[dnode] (3) at (3,0) {};
    \node[dnode,label=below:$n-1$] (4) at (4,0) {};
    \node[dnode,label=below:$n$] (5) at (5,0) {};

    \path (2) edge[tedge] (1)
          (2) edge[sedge,dashed] (3)
          (3) edge[sedge] (4)
          (4) edge[dedge] (5)
          ;
\end{tikzpicture}

%\begin{tikzpicture}
%
%    \draw (-1,0) node[anchor=east]  {$X_n$};
%
%\matrix (dynk) [matrix of nodes,nodes in empty cells,anchor=west,column sep=6mm]  at (-0.7,0) {
%%    \node[anchor=east] {$X_n$}; & 
%    \node[dnode,label=below:$1$] {}; p1 &
%    \node[dnode,label=below:$2$] {}; p2 & &
%    \node[dnode] {}; p3 &
%    \node[dnode,label=below:$n-1$] {}; p4 &
%    \node[dnode,label=below:$n$] {}; p5 \\
%  };
%
%    \path (dynk-1-2) edge[tedge] (dynk-1-1)
%          (dynk-1-2) edge[sedge,dashed] (dynk-1-3)
%          (dynk-1-3) edge[sedge] (dynk-1-4)
%          (dynk-1-4) edge[dedge] (dynk-1-5)
%          ;
%
%\end{tikzpicture}

\tikz \graph [grow right, branch right] {
  A -> B -> C;
};

\subsection*{Deal with scaling}

\begin{tikzpicture}[scale=0.5]
    \draw (-1,0) node[anchor=east]  {$X_n$};

    \node[dnode,label=below:$1$] (1) at (0,0) {};
    \node[dnode,label=below:$2$] (2) at (1,0) {};
    \node[dnode,label=below:$n-2$] (3) at (3,0) {};
    \node[dnode,label=below:$n-1$] (4) at (4,0) {};
    \node[dnode,label=below:$n$] (5) at (5,0) {};

    \path (1) edge[tedge] (2)
          (2) edge[sedge,middlearrow] (3)
          (3) edge[sedge] (4)
          (4) edge[dedge] (5)
          ;
\end{tikzpicture}


\begin{tikzpicture}[scale=1.0]
    \draw (-1,0) node[anchor=east]  {$X_n$};

    \node[dnode,label=below:$1$] (1) at (0,0) {};
    \node[dnode,label=below:$2$] (2) at (1,0) {};
    \node[dnode,label=below:$n-2$] (3) at (3,0) {};
    \node[dnode,label=below:$n-1$] (4) at (4,0) {};
    \node[dnode,label=below:$n$] (5) at (5,0) {};

    \path (1) edge[tedge] (2)
          (2) edge[sedge,middlearrow] (3)
          (3) edge[sedge] (4)
          (4) edge[dedge] (5)
          ;
\end{tikzpicture}


\begin{tikzpicture}[scale=2,thick]
    \draw (-1,0) node[anchor=east]  {$X_n$};

    \node[dnode,label=below:$1$] (1) at (0,0) {};
    \node[dnode,label=below:$2$] (2) at (1,0) {};
    \node[dnode,label=below:$n-2$] (3) at (3,0) {};
    \node[dnode,label=below:$n-1$] (4) at (4,0) {};
    \node[dnode,label=below:$n$] (5) at (5,0) {};

    \path (1) edge[tedge] (2)
          (2) edge[sedge,middlearrow] (3)
          (3) edge[sedge] (4)
          (4) edge[dedge] (5)
          ;

\end{tikzpicture}



\subsection*{Alternate ways of drawing triple edges}


\tikzset{
    triple/.style args={[#1] in [#2] in [#3]}{
        #1,preaction={preaction={draw,#3},draw,#2}
    }
}        
\begin{tikzpicture}
  \draw[triple={[line width=1mm,red] in
      [line width=2mm,white] in
      [line width=4mm,black]}] (3,3) to (5,5);

    \draw (-1,0) node[anchor=east]  {$X_n$};

    %\node[label=below:$1$] (1) at (0,0) {};
    %\node[label=below:$2$] (2) at (1,1) {};
    \node[dnode,label=below:$1$] (1) at (0,0) {};
    \node[dnode,label=below:$2$] (2) at (1,1) {};

%  \draw[triple={[line width=1mm,red] in
%      [line width=2mm,white] in
%      [line width=4mm,black]}] (1) to (2);

%  \path (1) edge[triple={[line width=1mm,red] in
%      [line width=2mm,white] in
%      [line width=4mm,black]}] (2);

  \path (1) edge[line width=3mm] (2);


\end{tikzpicture}

\end{document}
